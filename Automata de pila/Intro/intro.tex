\chapter{Introducción}
Me complace presentarle el informe de mi práctica individual sobre la implementación de un autómata de pila (PDA, por sus siglas en inglés) para reconocer el lenguaje libre de contexto $[$0$^$n 1$^$n | n >= 1$]$. En esta práctica, se nos proporcionaron las instrucciones necesarias para resolver el problema planteado y se nos solicitaron ciertas características específicas para el programa.\newline

El objetivo principal de esta práctica fue diseñar y desarrollar un autómata de pila que pudiera reconocer correctamente las cadenas pertenecientes al lenguaje $[$0$^$n 1$^$n | n >= 1$]$. Además, se nos pidió implementar ciertas funcionalidades adicionales para enriquecer la experiencia del usuario y facilitar la evaluación del autómata.\newline

Las características solicitadas para el programa fueron las siguientes:\newline

\begin{enumerate}
\item La posibilidad de que el usuario ingrese la cadena manualmente o que sea generada automáticamente. En el segundo caso, se estableció como requisito que la cadena no pudiera exceder los 100,000 caracteres.\newline

\item La generación de un archivo y la visualización en pantalla de las descripciones instantáneas (IDs) correspondientes a la evaluación del autómata. Esta función permitirá analizar paso a paso el proceso de reconocimiento de la cadena ingresada.\newline

\item La animación del autómata de pila, siempre y cuando la cadena tenga una longitud menor o igual a 10 caracteres. Esta característica adicional brindará una representación visual más interactiva del funcionamiento del autómata.\newline

\item La inclusión de pantallas del programa en ejecución en el informe, mostrando todas las características solicitadas. Estas capturas de pantalla servirán para ilustrar y respaldar los resultados obtenidos durante la implementación.\newline

\item La presentación del código de la implementación en formato LaTeX, en lugar de utilizar imágenes. Esto permitirá una visualización más clara y facilitará la revisión del código.\newline

\end{enumerate}
    
A lo largo de este informe, se detallarán los pasos seguidos para cumplir con los requisitos mencionados, incluyendo el diseño del autómata, la implementación del programa en Python, el registro de las descripciones instantáneas, la animación cuando corresponda y la generación del informe en LaTeX.