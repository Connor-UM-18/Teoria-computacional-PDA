\chapter{Anexos}
\lstset{
    language=C,
    basicstyle=\ttfamily\small\color{black},
    numbers=left,
    numberstyle=\tiny,
    stepnumber=1,
    numbersep=8pt,
    backgroundcolor=\color{white},
    showspaces=false,
    showstringspaces=false,
    showtabs=false,
    frame=single,
    rulecolor=\color{magenta},
    tabsize=2,
    captionpos=b,
    breaklines=true,
    breakatwhitespace=false,
    title=\lstname,
    escapeinside={\%*}{*)},
    keywordstyle=\color{blue},
    commentstyle=\color{red},
    stringstyle=\color{orange},
    morecomment=[l][\color{red}]{\#},
    otherkeywords={=,!,<,>,*,+,-,&,|,^,~},
    numbers=left,                   % Coloca los números de línea a la izquierda
    numberstyle=\tiny\color{black}, % Estilo de los números de línea
    stepnumber=1,    % Incremento en el que se muestran los números de línea
    numbersep=8pt
}
\section{PDA.py}
Se presenta el código implementado para la solución al problema con extensión .py . \newline
\\
\begin{lstlisting}
#Teoria de la computacion
#Automata de pila
#Alumno: Connor Urbano Mendoza

import random
from PIL import Image, ImageDraw,ImageFont
import sys

# Tamaño de la imagen y tamaño del cuadrado
image_size = (400, 400)
square_size = 100

# Crear una imagen en blanco
image = Image.new("RGB", image_size, "white")
draw = ImageDraw.Draw(image)



# Definir el PDA con sus transiciones
PDA = {
    ('q', '0', 'Z0'): [('q', 'X')],
    ('q', '0', 'X'): [('q', 'X')],
    ('q', '1', 'X'): [('p', '')],
    ('p', '1', 'X'): [('p', 'Z0')],
    ('p', '', 'Z0'): [('f', 'Z0')]
}

#Funcion de recorrido sin graficar
def proceso_recorrido(cadena,estado,i):
    ruta_archivo = "C:\\Users\\soyco\\OneDrive\\Documents\\ESCOM\\sem4\\Teoria\\P2\\Prog4\\output\\trancisiones.txt"
    with open(ruta_archivo, "w") as archivo:
        estado_actual=estado
        
        while i < (len(cadena)+1):
            if i == (len(cadena)):
                symbol=''
            else:
                symbol = cadena[i]
            cadena_pila = ''.join(pila)

            cima_de_pila = pila[-1]
            archivo.write("("+estado_actual+","+symbol+","+cadena_pila+") |-")
            if (estado_actual, symbol, cima_de_pila) in PDA:
                transiciones = PDA[(estado_actual, symbol, cima_de_pila)]
                estado_siguiente = transiciones[0][0]  
                letra_apuntada = transiciones[0][1]
                if(letra_apuntada == ''):
                    cima_de_pila = pila.pop()
                elif(letra_apuntada == 'Z0'):
                    cima_de_pila = pila.pop()
                else:
                    pila.append(letra_apuntada)
                    
                estado_actual=estado_siguiente
                i += 1
            else:
                archivo.write("Error")
                return False
        archivo.write("("+estado_actual+","+symbol+","+cadena_pila+")")    
        return True    
#Funcion de recorrido para graficar
def proceso_recorrido2(cadena,estado,i):
    
    ruta_archivo = "C:\\Users\\soyco\\OneDrive\\Documents\\ESCOM\\sem4\\Teoria\\P2\\Prog4\\output\\trancisiones.txt"
    with open(ruta_archivo, "w") as archivo:
        estado_actual=estado
        cadena_grafica = cadena
            
        while i < (len(cadena)+1):

            # Coordenadas del rectángulo
            rect_width = 40
            rect_height = 140
            x1 = ((image_size[0] - rect_width) // 2)
            desplazamientoa_abajo=60
            y1 = ((image_size[1] - rect_height) // 2) +desplazamientoa_abajo
            x2 = x1 + rect_width
            y2 = y1 + rect_height
            
            # Dibujar el rectángulo en la imagen
            draw.rectangle([(x1, y1), (x2, y2)], outline="black")
            #----------Dibujamos apartados-------------------------
            # Dibujamos apartados de cadena ingresada
            text = "Cadena ingresada:"
            font = ImageFont.truetype("arial.ttf", 16)

            # Dibujar el texto en la imagen
            draw.text((50, 40), text, fill="black", font=font)
            
            # Dibujar ecadenal texto en la imagen 
            draw.text((200, 40), cadena, fill="black", font=font) 
            
            
            # Dibujamos apartados de por ingresar
            text = "Por ingresar:"

            # Dibujar el texto en la imagen
            draw.text((50, 60), text, fill="black", font=font)
            
            #Dibujamos primer caracter
            j=0
            
            cadena_grafica = cadena_grafica[1:]  # Obtener la subcadena sin la primera letra
            draw.text((209, 61), cadena_grafica, fill="black", font=font)
            # Dibujamos apartados de entrada
            text = "Entrada:"
            font2 = ImageFont.truetype("arial.ttf", 20)

            # Dibujar el texto en la imagen
            draw.text((50, 90), text, fill="black", font=font2)
            
            #Coordenada de incremento para cruces
            cross_y = (((y1 + y2) // 2)+50) - ((i+1)*19)
            #Dibujamos caso base
            draw.text((189, (((y1 + y2) // 2)+50)), "Z0", fill="blue", font=font2)
            
            
            if i == (len(cadena)):
                symbol=''
                #Dibujamos primer caracter y la entrada
                # Dibujar el texto en la imagen
                draw.text((140, 91), "' '", fill="green", font=font2)
                draw.text((200, 61), "Vacio (E)", fill="red", font=font)
                #Dibujamos felcha de conversion
                draw.text((155, 91), "—>", fill="black", font=font2)
            else:
                symbol = cadena[i]
                number=int(symbol)
                # Dibujar el texto en la imagen
                draw.text((140, 91), symbol, fill="red", font=font2)
                draw.text((200, 61), symbol, fill="red", font=font)
                #Dibujamos felcha de conversion
                draw.text((155, 91), "—>", fill="black", font=font2)
                
            cadena_pila = ''.join(pila)
            cima_de_pila = pila[-1]
            
            archivo.write("("+estado_actual+","+symbol+","+cadena_pila+") |-")
            if (estado_actual, symbol, cima_de_pila) in PDA:
                transiciones = PDA[(estado_actual, symbol, cima_de_pila)]
                estado_siguiente = transiciones[0][0]  
                letra_apuntada = transiciones[0][1]
                if(letra_apuntada == ''):
                    cima_de_pila = pila.pop()
                    #Dibujamos letra
                    draw.text((197, 91), "-X", fill="blue", font=font2)
                    draw.text((197, 125), "^", fill="red", font=font2)
                    draw.text((199, 141), "|", fill="red", font=font2)
                    if len(cadena) % 2 != 0:
                        cross_y = ((((y1 + y2) // 2)+50) - ((i+1)*19)-19)
                    else:
                        cross_y = (((y1 + y2) // 2)+50) - ((i+1)*19)
                    # Coordenadas de la última cruz agregada
                    last_cross_x = 200
                    
                    last_cross_y = cross_y + ((38*i)-(19*len(cadena)))
                    last_cross_size = 12

                    # Guardar las coordenadas y el tamaño de la última cruz
                    last_cross_coords = [((last_cross_x - last_cross_size)-1, last_cross_y + 21), ((last_cross_x + last_cross_size)+1, (last_cross_y + 37)+1)]
                    # Eliminar la última cruz
                    draw.rectangle(last_cross_coords, outline="white", fill="white")
                    
                elif(letra_apuntada == 'Z0' and number==1 and symbol!=''):
                    cima_de_pila = pila.pop()
                    #Dibujamos letra
                    draw.text((197, 91), "-X", fill="blue", font=font2)
                    draw.text((197, 125), "^", fill="red", font=font2)
                    draw.text((199, 141), "|", fill="red", font=font2)
                    if len(cadena) % 2 != 0:
                        cross_y = ((((y1 + y2) // 2)+50) - ((i+1)*19)-19)
                    else:
                        cross_y = (((y1 + y2) // 2)+50) - ((i+1)*19)
                    # Coordenadas de la última cruz agregada
                    last_cross_x = 200
                    #print(cross_y)
                    last_cross_y = cross_y + ((38*i)-(19*len(cadena)))
                    last_cross_size = 12

                    # Guardar las coordenadas y el tamaño de la última cruz
                    last_cross_coords = [((last_cross_x - last_cross_size)-1, last_cross_y + 21), ((last_cross_x + last_cross_size)+1, (last_cross_y + 37)+1)]
                    # Eliminar la última cruz
                    draw.rectangle(last_cross_coords, outline="white", fill="white")
                elif(letra_apuntada == 'Z0' and symbol==''):
                    cima_de_pila = pila.pop()
                    #Dibujamos letra
                    draw.text((197, 91), "Z0", fill="blue", font=font2)
                    draw.text((197, 125), "^", fill="red", font=font2)
                    draw.text((199, 141), "|", fill="red", font=font2)
                    # Coordenadas de la última cruz agregada
                    last_cross_x = 200
                    #print(cross_y)
                    last_cross_y = cross_y + ((38*i)-(19*len(cadena)))
                    last_cross_size = 12

                    # Guardar las coordenadas y el tamaño de la última cruz
                    last_cross_coords = [((last_cross_x - last_cross_size)-1, last_cross_y + 21), ((last_cross_x + last_cross_size)+1, (last_cross_y + 37)+1)]
                    # Eliminar la última cruz
                    draw.rectangle(last_cross_coords, outline="white", fill="white")
                else:
                    pila.append(letra_apuntada)
                    #Dibujamos letra
                    draw.text((197, 91), letra_apuntada, fill="blue", font=font2)
                    draw.text((201, 121), "|", fill="green", font=font2)
                    draw.text((197, 141), "V", fill="green", font=font2)
                    #Dibujamos cruz
                    draw.text((194, cross_y), "X", fill="black", font=font2)
                    
                estado_actual=estado_siguiente
                
                # Guardar la imagen en la ruta especificada
                image.save("C:\\Users\\soyco\\OneDrive\\Documents\\ESCOM\\sem4\\Teoria\\P2\\Prog4\\output\\animacion.png")
                if sys.stdin.isatty():
                    print("Presiona Enter para continuar...")
                    sys.stdin.readline()
                else:
                    input("Presiona Enter para continuar...")
                i += 1
            else:
                archivo.write("Error")
                #Dibujamos letra
                # Cadena invalida
                draw.text((200, 40), cadena, fill="red", font=font) 
                draw.text((197, 91), "ERROR", fill="blue", font=font2)
                
                image.save("C:\\Users\\soyco\\OneDrive\\Documents\\ESCOM\\sem4\\Teoria\\P2\\Prog4\\output\\animacion.png")
                if sys.stdin.isatty():
                    print("Presiona Enter para continuar...")
                    sys.stdin.readline()
                else:
                    input("Presiona Enter para continuar...")
                return False
            
            # Aquí se realiza el borrado del contenido de la imagen
            draw.rectangle([(11, 11), (389, 180)], fill="white")
            image.save("C:\\Users\\soyco\\OneDrive\\Documents\\ESCOM\\sem4\\Teoria\\P2\\Prog4\\output\\animacion.png")

            
        archivo.write("("+estado_actual+","+symbol+","+cadena_pila+")")    
        
        # Coordenadas del rectángulo
        rect_width = 40
        rect_height = 140
        x1 = ((image_size[0] - rect_width) // 2)
        desplazamientoa_abajo=60
        y1 = ((image_size[1] - rect_height) // 2) +desplazamientoa_abajo
        x2 = x1 + rect_width
        y2 = y1 + rect_height
        
        # Dibujar el rectángulo en la imagen
        draw.rectangle([(x1, y1), (x2, y2)], outline="black")
        
        # Dibujamos apartados de cadena ingresada
        text = "Cadena ingresada:"
        font = ImageFont.truetype("arial.ttf", 16)

        # Dibujar el texto en la imagen
        draw.text((50, 40), text, fill="black", font=font)
        
        # Dibujar ecadenal texto en la imagen 
        draw.text((200, 40), cadena, fill="green", font=font) 
        
        text = "Estado Final"

        # Dibujar el texto en la imagen
        draw.text((50, 60), text, fill="black", font=font)
        
        # Dibujamos apartados de entrada
        text = "Ya sacamos a Z0"
        # Dibujar el texto en la imagen
        draw.text((50, 92), text, fill="black", font=font)
        #Dibujamos felcha de conversion
        draw.text((178, 91), "——>", fill="black", font=font2)
                
        
        draw.text((235, 91), "Z0", fill="blue", font=font2)
        
        image.save("C:\\Users\\soyco\\OneDrive\\Documents\\ESCOM\\sem4\\Teoria\\P2\\Prog4\\output\\animacion.png")
        if sys.stdin.isatty():
            sys.stdin.readline()
        else:
            input("Presiona Enter para continuar...")
        return True    

def generar_cadena_aleatoria(tamanio):
    mitad = tamanio // 2
    ceros = '0' * mitad
    unos = '1' * mitad
    if tamanio % 2 != 0:
        ceros += random.choice(['0', '1'])
    cadena = ceros + unos
    return cadena
    
    
#Main
pila = []

elemento_superior = 'Z0'
pila.append(elemento_superior)
estado = 'q'
i = 0

cadena = input("Ingresa una cadena (o presiona Enter para generar una cadena aleatoria): ")

if not cadena:
    tamanio = random.randint(1, 100000)
    print(tamanio)
    cadena = generar_cadena_aleatoria(tamanio)
    resultado = proceso_recorrido(cadena, estado, i)

    if resultado:
        print("La cadena es válida.")
    else:
        print("La cadena no es válida.")
    
else:
    resultado = proceso_recorrido2(cadena, estado, i)

    if resultado:
        print("La cadena es válida.")
    else:
        print("La cadena no es válida.")
    
\end{lstlisting}

Se presenta el código LaTeX de este archivo mediante el siguiente link:\newline
\\
Link overleaf: $"$\url{https://www.overleaf.com/read/tpnhwgwdrprq}$"$\newline
Link github: $"$\url{https://github.com/Connor-UM-18/Teoria-computacional-PDA.git}$"$\newline
